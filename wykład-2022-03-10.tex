\documentclass{article}

\usepackage[utf8]{inputenc}
\usepackage{amsfonts}
\usepackage{amsmath}
\usepackage{polski}

\newcommand\CC{\mathbb{C}}

\begin{document}

\section{Iloczyn tensorowy macierzy}

$$
\begin{aligned}
A = [ a_{i,j} ] \in M_{r x s}(\CC) \\
B = [ b_{i,j} ] \in M_{t x u}(\CC)
\end{aligned}
$$

iloczynem tensorowym macierzy $A$ i $B$ nazywamy macierz $ A \otimes B = M_{rt \times su}(\CC) $


$$
A \otimes B = \begin{bmatrix}
	a_{11} & B a_{12} & a_{13} B & \dots & a_{1s} B \\
	a_{21} & B a_{22} & a_{23} B & \dots & a_{2s} B \\
	\vdots                                          \\
	a_{r1} & B a_{r2} & a_{r3} B & \dots & a_{rs} B \\
\end{bmatrix}
$$

Przykład:

$$
\begin{bmatrix}1 \\ 0\end{bmatrix} \otimes \begin{bmatrix}0 \\ 1\end{bmatrix} =
	\begin{bmatrix}1 \begin{bmatrix} 0 \\ 1 \end{bmatrix} \\ 0 \begin{bmatrix} 0 \\ 1 \end{bmatrix} \end{bmatrix} = \begin{bmatrix} 0 \\ 1 \\ 0 \\ 0 \end{bmatrix}
$$

\subsection{Własności iloczynu tensorowego}

Iloczyn tensorowy tworzy półgrupę.

\begin{description}
	\item $ A \otimes (B + C) = A \otimes B + A \otimes C $
	\item $ (B+C) \otimes A = B \otimes A + C \otimes A $
	\item $ A \otimes (B \otimes C) = (A \otimes B) \otimes C $
	\item $ (\alpha A) \otimes B = A \otimes (\alpha B) = \alpha (A \otimes B) $
	\item $ (A \otimes B)(C \otimes D) = AC \otimes BD $
\end{description}

\section{Postulaty mechaniki kwantowej}


	\subsection{Postulat 1, postulat układu}
	Każdemu układowi kwantowemu można przypisać skończenie wymiarową $\CC^n$ przestrzeń Hilberta, w której określa się kwantowo-mechaniczną teorię tego układu. Układ ten jest opisywany przez unormowany wektor stanu. (Jeden stan wyrażony jest przez jeden konkretny wektor).

		$$ |Y\rangle = \alpha |0\rangle + \beta| 1\rangle \in \CC^2 $$ dla $ |\alpha|^2 + |\beta|^2 = 1 $

	\subsection{Postulat 2, postulat ewolucji układu} \label{postulat2}
	Ewolucja układu kwantowego $|\psi\rangle$ jest opisywana przez przekształcenia unitarne przestrzenii $\CC^2$. To znaczy, że wektor $|\psi\rangle$ w czasie $t_1$ jest związany z wektorem |Y'> w czasie $t_2$ operatorem unitarnym $U$.

		$$ |\psi'\rangle = U|\psi\rangle $$

		Algorytm kwantowy to sekwencja przekształceń unitarnych.

	\subsection{Postulat 3, postulat iloczynu tensorowego} Jeżeli $\CC^n$ i $\CC^m$ są przestrzeniami stanów dwóch niezależnych i nierozróżnialnych układów kwantowych, to przestrzeń stanów obu tych układów branych jako całość jest iloczynem tensorowym tych przestrzeni.

		$$
	 |00\rangle = |0\rangle|0\rangle	= |0\rangle \otimes |0\rangle = \begin{bmatrix}1 \\ 0\end{bmatrix} \otimes \begin{bmatrix}1 \\ 0\end{bmatrix} = \begin{bmatrix}1 \\ 0 \\ 0 \\ 0\end{bmatrix}
		$$

		Dla kubitu $ |\psi\rangle \in \CC^4 $ mamy:
		$$ |\psi\rangle = \alpha_1|00\rangle + \alpha_2|01\rangle + \alpha_3|10\rangle + \alpha_4|11\rangle $$ gdzie $ \sum_\alpha |\alpha|^2 = 1 $

		Nie każdy stan można przedstawić jako iloczyn tensorowy np. $ \frac{1}{\sqrt{2}}|00\rangle + \frac{1}{\sqrt{2}} |11\rangle \in \CC^4 $. (jest to układ nierozkładalny, splątanie kwantowe)

		$$
		(\alpha_1 |0\rangle + \beta_1|1\rangle) \otimes (\alpha_2|0\rangle + \beta_2|1\rangle) =
		(\alpha_1 |0\rangle + \beta_1|1\rangle) \otimes \alpha_2|0\rangle + (\alpha_1 |0\rangle + \beta_1|1\rangle) \otimes \beta_2|1\rangle = \alpha_1\alpha_2|00\rangle + \beta_1\alpha_2|10\rangle + \alpha_1\beta_2|01\rangle + \beta_1\beta_2|11\rangle
		$$ wynika z tego układ równań (sprzeczny).
		$$
		\begin{aligned}
			\alpha_1\alpha_2 & = \frac{1}{\sqrt{2}} \\
			\beta_1\beta_2 & = \frac{1}{\sqrt{2}} \\
			\beta_1\alpha_2 & = 0 \\
			\alpha_2\beta_2 & = 0
		\end{aligned}
		$$

	\subsection{Postulat 4, postulat pomiaru} Pomiar kwantowy w bazie obliczeniowej jest opisywany przez zbiór $\{ M_m \}$ (macierzy) (hermitowskie operatory projekcji). Operatory te działają w przypisanej przestrzenii stanów mierzonego układu. Indeks $m$ jest wynikiem pomiaru.

		Jeśli $|\psi\rangle$ jest mierzonym stanem, to $p(m)$ - prawdopodobieństwo, że wynikiem pomiaru będzie $m$ wynosi $p(m) = \langle\psi M_m |\psi\rangle $, a stan $|\psi\rangle$ po pomiarze znajdzie się w stanie $$ |\psi'\rangle = \frac{M_m|\psi\rangle}{\sqrt{p(m)}} $$.

		Przykład: Dla $|\psi\rangle = \frac{1}{\sqrt{2}} |0\rangle + \frac{1}{\sqrt{2}}|1\rangle$ mamy
		$$ M_0 = |0\rangle \langle0| = \begin{bmatrix} 1 & 0 \\ 0 & 0 \end{bmatrix} $$
		$$ M_1 = |1\rangle \langle1| = \begin{bmatrix} 0 & 0 \\ 0 & 1 \end{bmatrix} $$

		Pomiar to	$ \{ M_0, M_1 \} $

		$$ p(0) = \begin{bmatrix}  \frac{1}{\sqrt{2}} &  \frac{1}{\sqrt{2}} \end{bmatrix} \begin{bmatrix} 1 & 0 \\ 0 & 0 \end{bmatrix} \begin{bmatrix} \frac{1}{\sqrt{2}} \\ \frac{1}{\sqrt{2}}  \end{bmatrix} = \frac{1}{2}$$

			$$ p(1) = \begin{bmatrix}  \frac{1}{\sqrt{2}} &  \frac{1}{\sqrt{2}} \end{bmatrix} \begin{bmatrix} 0 & 0 \\ 0 & 1 \end{bmatrix} \begin{bmatrix} \frac{1}{\sqrt{2}} \\ \frac{1}{\sqrt{2}}  \end{bmatrix} = \frac{1}{2} $$

			$$ |\psi'\rangle = \frac{M_0|\psi\rangle}{\sqrt{\frac{1}{2}}} = 1|0\rangle $$

		Dla kubitu: \begin{description}
			\item[Pomiar $ p(0) = |\alpha|^2 $] tworzy $ |\psi'\rangle = \frac{\alpha}{|\alpha|}|0\rangle $
			\item[Pomiar $ p(1) = |\beta|^2 $] tworzy $ |\psi'\rangle = \frac{\beta}{|\beta|}|1\rangle $
		\end{description}

\section{Bramki kwantowe}

Przekształcenia unitarne z \ref{postulat2} to bramki algorytmów kwantowych.

\subsection{Bramka hadamarda}

Rysunek

\subsection{Bramka kontrolowanej negacji}

Rysunek

\end{document}
